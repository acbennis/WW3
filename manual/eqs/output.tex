\vssub
\subsection{~Output parameters} \label{sub:outpars}
\vssub

The wave model can output parameters on the geographical grid of the model that can be integrated parameters or 
parameters as a function of frequency, with one grid for each frequency. All parameters that are a function of frequency (e.g.  \textbf{EF} or \textbf{USF}) require the setting 
of specific namelist parameters in the {\F OUTS} namelist defined in ww3\_grid.inp or  ww3\_grid.nml . This is to reduce the memory use
if these parameters are not needed. 

Below, a brief definition of output field parameters is provided. A table
with definitions may be found in the sample {\code ww3\_shel.inp} file,
in \para\ref{sec:ww3shel}. That input file also provides a list of flags  
indicating if output parameters are available in different field
output file types (ASCII, grib, igrads, NetCDF). 
For any details on how these parameters are computed, the user may read the code of the  {\code w3iogo} routine, in the {\code w3iogomd.ftn} module. 

Selection of field outputs in  {\code ww3\_shel.inp} is most easily performed by providing a list of the 
requested parameters, for example, {\textbf HS DIR SPR}  will request the calculation of significant wave height, mean direction and directional spread. These will thus be stored in the {\code out\_grd.XX} file and can be post-processed, for example in NetCDF using   {\code ww3\_ouf}. Examples are given in \para\ref{sec:ww3multi} and
\para\ref{sec:ww3ounf}. The names for these namelists are the bold names below, for 
example \textbf{HS}. 

All parameters listed below are available in ASCII and NetCDF output files. If selected output file types are grads or grib, some parameters may not be available. Availability (or not) is identified
in the first two columns in the field output parameter table within the example input file in  
\para\ref{sec:ww3shel}. That table also identifies, for all parameters,
the internal WAVEWATCH III code tags, the output tags (names used is ASCII 
file extensions, NetCDF variable names and namelist-based selection (see 
also \para\ref{sec:ww3ounf}), and the long parameter name/definition.

Finally we note that in all definitions the frequency is the \emph{relative} frequency. Thus, in the presence of currents, these can only be compared to drifting measurement data or data obtained in wavenumber and converted to frequency. Comparison to fixed instrument data requires the use of the full spectrum and proper conversion to the fixed reference frame.

\begin{list}{\Roman{outgrps})\hfill}
            {\usecounter{outgrps} \leftmargin 7mm \labelwidth 5mm
             \rightmargin 0mm \itemsep \baselineskip \parsep 0mm}

\item {Forcing fields}

\begin{list}{\arabic{outpars})\hfill}
            {\usecounter{outpars} \leftmargin 8mm \labelwidth 7mm
             \rightmargin 0mm \itemsep 0mm \parsep 0mm}
\item \textbf{DPT} The mean water depth (m). This includes varying water levels. 
\item \textbf{CUR} The mean current velocity (vector, m/s).
\item \textbf{WND} The mean wind speed (vector, m/s). This wind speed is always the
      speed as input to the model, i.e., is not corrected for the current
      speed.
\item \textbf{AST} The air-sea temperature difference ($^\circ$C).
\item \textbf{WLV} Water level.
\item \textbf{ICE} Ice concentration.
\item \textbf{IBG} Wave attenuation due to icebergs: this parameter is the inverse of the
  e-folding scale associated to the loss of wave energy in a field of small
  icebergs \citep{art:Aea11}.
\item \textbf{D50} Sediment median grain size ($D_{50}$). 
\item \textbf{IC1} Ice thickness. 
\item \textbf{IC5} Maximum ice flow diameter, $D_{\max}$. 
\end{list}

\item{Standard mean wave parameters}

\begin{list}{\arabic{outpars})\hfill}
            {\usecounter{outpars} \leftmargin 8mm \labelwidth 7mm
             \rightmargin 0mm \itemsep 0mm \parsep 0mm}
\item \textbf{HS} Significant wave height (m) [see Eq.~(\ref{eq:etot})]
      \begin{equation} H_s = 4 \sqrt{E} \: . \label{eq:Hs} \end{equation}
\item \textbf{LM} Mean wave length (m) [see Eq.~(\ref{eq:zbar})]
      \begin{equation} L_m = 2\pi \overline{k^{-1}}
      \: . \label{eq:Lm} \end{equation}
\item \textbf{T02} Mean wave period (s) [see Eq.~(\ref{eq:zbar})]
      \begin{equation} T_{m02} = 2\pi /\sqrt{\overline{\sigma^{2}}}
      \: . \label{eq:Tm02} \end{equation}
\item \textbf{T0M1} Mean wave period (s) [see Eq.~(\ref{eq:zbar})]
      \begin{equation} T_{m0,-1} = 2\pi \overline{\sigma^{-1}}
      \: . \label{eq:Tm0m1} \end{equation}
\item \textbf{T01} Mean wave period (s) [see Eq.~(\ref{eq:zbar})]
      \begin{equation} T_{m0,1} = 2\pi /\overline{\sigma}
      \: . \label{eq:Tm} \end{equation}
\item \textbf{FP} Peak frequency (Hz), calculated from the one-dimensional frequency
      spectrum using a parabolic fit around the discrete peak.
\item \textbf{DIR} Mean wave direction (degr., meteorological convention)
      \begin{equation} \theta_m = \mbox{atan} \left ( \frac{b}{a} \right )
      \: , \label{eq:theta_m} \end{equation} \begin{equation}
      a = \int_0^{2\pi} \int_0^\infty \cos(\theta) F(\sigma,\theta) \:
      d\sigma \: d\theta \: , \end{equation} \begin{equation}
      b = \int_0^{2\pi} \int_0^\infty \sin(\theta) F(\sigma,\theta) \:
      d\sigma \: d\theta \: . \end{equation}
\item \textbf{SPR} Mean directional spread \citep[degr.;][]{art:KVH88}
      \begin{equation} \sigma_\theta = \left [ 2 \left \{ 1 - \left (
      \frac{a^2+b^2}{E^2} \right )^{1/2} \right \} \right ]^{1/2}
      \: , \label{eq:sig_th} \end{equation}
\item \textbf{DP} Peak direction (degr.), defined like the mean direction, using the
      frequency/wavenumber bin containing of the spectrum $F(k)$ that
      contains the peak frequency only.
\item \textbf{HIG} Infragravity height.
\item \textbf{MXE}   Max surface elev (Space-time extreme, STE)
\item \textbf{MXES}  St Dev of max surface elev (STE)
\item \textbf{MXH}  Max wave height (STE)
\item \textbf{MXHC}  Max wave height from crest (STE)
\item \textbf{SDMH}  St Dev of MXC (STE)
\item \textbf{SDMHC} St Dev of MXHC (STE)
\item \textbf{WBT}   Dominant wave breaking probability $b_T$ (\ref{eq:bt})
\end{list}

\item{Spectral parameters (first 5 moments and wavenumbers). 

All these parameters are a function 
of frequency and thus these are three-dimensional arrays. Because of the large memory use, the 
computation of these parameters requires the activation of switches in the {\F OUTS} namelist.}

\begin{list}{\arabic{outpars})\hfill}
            {\usecounter{outpars} \leftmargin 8mm \labelwidth 7mm
             \rightmargin 0mm \itemsep 0mm \parsep 0mm}
   
\item \textbf{EF} Wave frequency spectrum (m$^2$/Hz)
      \begin{equation} E(f) = 2 \pi  \int F(\sigma,\theta) \mathrm{d} \theta 
      \: . \label{eq:Ef} \end{equation}
\item \textbf{TH1M} Mean direction for each frequency  \citep[degr.;][]{art:KVH88}
       \begin{equation} \theta_1 (f)= \mbox{atan} \left ( \frac{b_1(f)}{a_1(f)} \right )
      \: , \label{eq:theta_b} \end{equation} \begin{equation}
      a_1(f) = 2 \pi \int_0^{2\pi} \int_0^\infty \cos(\theta) F(\sigma,\theta) \:
       \mathrm{d}\theta \: , \end{equation} \begin{equation}
      b_1(f) = 2 \pi \int_0^{2\pi} \int_0^\infty \sin(\theta) F(\sigma,\theta) \:
       \mathrm{d}\theta \: . \end{equation}
\item \textbf{STH1M} First directional spread for each frequency  (degr.; )
      \begin{equation} \sigma_1(f) = \left [ 2 \left \{ 1 - \left (
      \frac{a_1(f)^2+b_1(f)^2}{E(f)^2} \right )^{1/2} \right \} \right ]^{1/2}
      \: , \label{eq:sig_th1} \end{equation}
\item \textbf{TH2M} Mean direction from $a_2$ and $b_2$ (degr.)
       \begin{equation} \theta_2 (f)= \mbox{atan} \left ( \frac{b_2(f)}{a_2(f)} \right )
      \: , \label{eq:theta_2} \end{equation} \begin{equation}
      a_2(f) = 2 \pi \int_0^{2\pi} \int_0^\infty \cos(2 \theta) F(\sigma,\theta) \:
       \mathrm{d}\theta \: , \end{equation} \begin{equation}
      b_2(f) = 2 \pi \int_0^{2\pi} \int_0^\infty \sin(2 \theta) F(\sigma,\theta) \:
       \mathrm{d}\theta \: . \end{equation}
\item \textbf{STH2M} Directional spreading from $a_2$ and $b_2$ (degr.)  
      \begin{equation} \sigma_2(f) = \left [ 0.5 \left \{ 1 - \left (
      \frac{a_2(f)^2+b_2(f)^2}{E(f)^2} \right )^{1/2} \right \} \right ]^{1/2}
      \: , \label{eq:sig_th1} \end{equation}
\item \textbf{WN}  Wavenumbers $k(\sigma)$ (rad/m)
      \begin{equation} \sigma^2 = g k \tanh (k D) 
      \: , \label{eq:k} \end{equation}
\end{list}

\item{Spectral partition parameters} \label{out:spec_part}

These output parameters are based on
partitioning of the spectrum into individual wave fields.

\begin{list}{\arabic{outpars})\hfill}
            {\usecounter{outpars} \leftmargin 8mm \labelwidth 7mm
             \rightmargin 5mm \itemsep 0mm \parsep 0mm}


\item \textbf{PHS}  Wave heights $H_s$ of partitions of the spectrum (see
      below). \label{out:first_part}
\item \textbf{PTP} Peak (relative) periods of partitions of the spectrum (parabolic fit).
\item \textbf{PLP} Peak wave lengths of partitions of the spectrum (from peak period).
\item \textbf{PSP} Mean direction of partitions of the spectrum.
\item Directional spread of partition of the spectrum
      Cf. Eq.~(\ref{eq:sig_th}).
\item \textbf{PWS} Wind sea fraction of partition of the spectrum.  The method of
\cite{art:HP01} is used, implemented as described in \cite{tol:Oahu07a}. With
this, a `wind sea fraction' $W$ is introduced
\begin{equation}
W = E^{-1}  \: E |_{U_p > c}  \:\:\: , \label{eq:wsf}
\end{equation}
where $E$ is the total spectral energy, and $E |_{U_p > c}$ is the energy in
the spectrum for which the projected wind speed $U_p$ is larger than the local
wave phase velocity $c = \sigma / k$. The latter defines an area in the
spectrum under the direct influence of the wind. To allow for nonlinear
interactions to shift this boundary to lower frequencies, and subsequently to
have fully grown wind seas inside this are, $U_p$ includes a multiplier
$C_{mult}$
\begin{equation}
U_p = C_{mult} U_{10} \cos ( \theta - \theta_w )  \:\:\: . \label{eq:Up}
\end{equation}
The multiplier can be set by the user. The default value is $C_{mult} = 1.7$.
\item \textbf{PDP} Peak direction of partitions of the spectrum.
\item \textbf{PQP} Goda peakedness of partitions of the spectrum.
\item \textbf{PPE} JONSWAP peak enhancement factor of partitions
      of the spectrum. It is a measure of amplification
      of the spectral peak to the peak of the corresponding
      \citeauthor{art:PM64} spectrum.
\item \textbf{PGW} Gaussian frequency width (Hz) of partitions of the spectrum.
      Least-square fit of a Gaussian spectrum to the one-dimensional spectral partition.
\item \textbf{PSW} Spectral width of partitions of the spectrum \citep{art:LH84}
\item \textbf{PTE} Wave energy period of partitions of the spectrum [see Eq.~(\ref{eq:Tm0m1})] 
\item \textbf{PT1} Mean wave period of partitions of the spectrum [see Eq.~(\ref{eq:Tm})] 
\item \textbf{PT2} Wave zero-upcrossing period of partitions of the spectrum [see Eq.~(\ref{eq:Tm02})]
\item \textbf{PEP} Wave peak spectral density of partitions
\item \textbf{TWS} Wind sea fraction of the entire spectrum.
\item \textbf{PNR} Number of partitions found in the spectrum. \label{out:last_part}
\end{list}


\item{Atmosphere-waves layer}
\begin{list}{\arabic{outpars})\hfill}
            {\usecounter{outpars} \leftmargin 8mm \labelwidth 7mm
             \rightmargin 0mm \itemsep 0mm \parsep 0mm}
\item \textbf{UST}  The friction velocity $u_\ast$ (scalar). Definition depends on
      selected source term parameterization (m/s). An alternative vector version
      of the stresses is available for research (requires user intervention in
      the code).
\item \textbf{CHA}  Charnock parameter for air-sea friction (without dimensions)
\item \textbf{CGE} Energy flux (W/m)
      \begin{equation} C_g E =  \rho_w g \overline{C_g} E
      \: . \label{eq:CgE} \end{equation}
\item \textbf{FAW} Wind to wave energy flux
\item \textbf{TAW} Net wave-supported stress (wind to wave momentum flux) 
\item \textbf{TWA} Negative part of the wave-supported stress
\item \textbf{WCC} Whitecap coverage (without dimensions)
\item \textbf{WCF} Wave to wind momentum flux 
\item \textbf{WCH} Whitecap mean thickness (m) 
\item \textbf{WCM} Mean breaking wave height (m) (NOT AVAILABLE YET)
%\item \textbf{PWS} Moment of whitecap distribution (?) (NOT PLUGGED YET)
\end{list}

\item{Wave-ocean layer}

\begin{list}{\arabic{outpars})\hfill}
            {\usecounter{outpars} \leftmargin 8mm \labelwidth 7mm
             \rightmargin 0mm \itemsep 0mm \parsep 0mm}
\item \textbf{SXY} Radiation stresses
      \begin{equation} S_{xx} = \rho_w g \int \!\!\!\! \int \left
        ( n - 0.5 + n \cos^2 \theta \right )  \: F(k,\theta) \: dk d\theta
      \: , \label{eq:Sxx} \end{equation}
      \begin{equation} S_{xy} =\rho_w g \int \!\!\!\! \int
        n \sin \theta \cos \theta  \: F(k,\theta) \: dk d\theta
      \: , \label{eq:Syy} \end{equation}
      \begin{equation} S_{yy} =\rho_w g \int \!\!\!\! \int \left
        ( n - 0.5 + n \sin^2 \theta \right )  \: F(k,\theta) \: dk d\theta
      \: , \label{eq:Sxy} \end{equation}
      where
      \begin{equation} n = \frac{1}{2} + \frac{kd}{\sinh 2kd}
      \: . \label{eq:n} \end{equation}
\item \textbf{TWO} Wave to ocean momentum flux
\item \textbf{BHD} Bernoulli head (m$^2$/s$^2$)
      \begin{equation} J =  g \int \!\!\!\! \int  \frac{k}
        {\sinh 2kd}  \: F(k,\theta) \: dk d\theta
      \: , \label{eq:BHD} \end{equation}
\item \textbf{FOC} Wave to ocean energy flux (W/m$^2$)
\item \textbf{TUS} Stokes volume transport (m$^2$/s)
     \begin{equation} (M^w_x,M^w_y) =  g \int \!\!\!\! \int  \frac{(k \cos(\theta),k \sin(\theta))}
        {\sigma}  \: F(k,\theta) \: dk d\theta
      \: , \label{eq:Mw} \end{equation}
\item \textbf{USS} Stokes drift at the sea surface (m/s)
     \begin{equation} (U_{ssx},U_{ssy}) =  \int \!\!\!\! \int  \sigma \cosh 2kd \frac{(k \cos(\theta),k \sin(\theta))}
        {\sinh^2 kd}  \: F(k,\theta) \: dk d\theta
      \: , \label{eq:Uss} \end{equation}
\item \textbf{P2S} Second order pressure variance (m$^2$) and peak period of this pressure (s) which contributes to acoustic and seismic noise, 
      \begin{equation} F_{p2D}(k=0) = \int_0^{\infty} \frac{4 \sigma}{C_g} \int_0^{\pi}  F(k,\theta) \:  F(k,\theta+\pi) \:  d\theta dk
      \: , \label{eq:sop} \end{equation}
\item \textbf{USF} Frequency spectrum of Stokes drift at the sea surface (m/s/Hz)
     \begin{equation} (U_{ssx}(f),U_{ssy}(f)) =  \int \sigma \cosh 2kd \frac{(k \cos(\theta),k \sin(\theta))}
        {\sinh^2 kd}  \: F(k,\theta) \frac{2 \pi}{C_g} d\theta
      \: , \label{eq:Ussf} \end{equation}
Note that US3D=1 must be set in the ww3\_grid.inp OUTS namelist in order to activate the 3D arrays needed for USF output.
\item \textbf{P2L} Frequency spectrum of the second order pressure (m$^2$s)  which contributes to acoustic and seismic noise,
      \begin{equation} F_{p2D}(k=0,f) =  \frac{2 \sigma}{\pi} \int_0^{\pi}  \frac{4 \pi^2}{C_g^2} F(k,\theta) \:  F(k,\theta+\pi) \:  d\theta
      \: . \label{eq:sopf} \end{equation}
\item \textbf{TWI}   Wave to sea ice stress
\item \textbf{FIC}   Wave to sea ice energy flux
\item \textbf{USP}   Surface Stokes drift partitioned into run-time defined frequency components (m/s),
      \begin{equation} (U_{ssp},V_{ssp})(n) =  \int \!\!\!\! \int_{k_n}  \sigma \cosh 2kd \frac{(k\
      \cos(\theta),k \sin(\theta))}
        {\sinh^2 kd}  \: F(k,\theta) \: dk d\theta
      \: . \label{eq:Ussp} \end{equation}
The number $n$ and the central wavenumbers for the partitions $k_n$ are `run-time' in the sense that they are defined in the model definition file generated from ww3\_grid.inp (see the sample input file for details on setting these quantities).  The wavenumber integral bounds for each $n$ partition divide the model wave spectra to accomodate the prescribed values of $k_n$.
\end{list}

\item{Wave-bottom layer}

\begin{list}{\arabic{outpars})\hfill}
            {\usecounter{outpars} \leftmargin 8mm \labelwidth 7mm
             \rightmargin 0mm \itemsep 0mm \parsep 0mm}

\item \textbf{ABR} Near-bottom rms excursion amplitude
      \begin{equation} a_{b,rms} = \left [ 2 \int \!\!\!\! \int
      \frac{1}{\sinh^2 kd} \: F(k,\theta) \: dk d\theta \right ] ^{1/2}
      \: . \label{eq:ab_rms} \end{equation}
\item \textbf{UBR} Near-bottom rms orbital velocity
      \begin{equation} u_{b,rms} = \left [ 2 \int \!\!\!\! \int
      \frac{\sigma^2}{\sinh^2 kd} \: F(k,\theta) \: dk d\theta \right ] ^{1/2}
      \: . \label{eq:ub_rms} \end{equation}
\item \textbf{BED} Bedform parameters: ripple height and directions (NOT TESTED YET)   
\item \textbf{FBB} Energy dissipation in WBBL 
\item \textbf{TBB} Momentum loss in WBBL 
\end{list}

\item{Spectrum parameters}

\begin{list}{\arabic{outpars})\hfill}
            {\usecounter{outpars} \leftmargin 8mm \labelwidth 7mm
             \rightmargin 0mm \itemsep 0mm \parsep 0mm}

\item \textbf{MSS} Mean square slopes in $u$ and $c$ directions (down-wave and cross-wave components of slopes variances). 
\item \textbf{MSC} Spectral tail level (without dimensions) 
\item \textbf{MSD} Direction of the maximum slope variance mss$_u$
\item \textbf{MCD} Spectral tail direction
\item \textbf{QP}  Peakedness parameter \citep{art:G70}
      \begin{equation} Q_p = \frac{2}{E^2} \int_0^{2\pi} \int_0^\infty
      \sigma\:F(\sigma,\theta)^2\:d\sigma\:d\theta \: \label{eq:qp}
      \end{equation}
\end{list}

\item{Numerical diagnostics }

\begin{list}{\arabic{outpars})\hfill}
            {\usecounter{outpars} \leftmargin 8mm \labelwidth 7mm
             \rightmargin 0mm \itemsep 0mm \parsep 0mm}
\item \textbf{DTD} Average time step in the source term integration (s).
\item \textbf{FC} Cut-off frequency $f_c$ (Hz, depends on parameterization of
      input and dissipation).
\item \textbf{CFX} Maximum CFL number for spatial advection 
\item \textbf{CFD} Maximum CFL number for angular advection 
\item \textbf{CFK} Maximum CFL number for wavenumber advection 
\end{list}

\item{User defined }

\begin{list}{\arabic{outpars})\hfill}
            {\usecounter{outpars} \leftmargin 8mm \labelwidth 7mm
             \rightmargin 0mm \itemsep 0mm \parsep 0mm}

\item \textbf{U1} Slot for user defined parameter (requires modification of code).
\item \textbf{U2} Idem.
\end{list}

\end{list}




%\vspace{\baselineskip}
%\centerline{\ldots NEEDS CLEAN UP BELOW HERE \ldots}
%\vspace{\baselineskip}

%THESE TWO PARAMETERS WILL BE PUT BACK. 
%Old output types \ref{out:old_fpw} and \ref{out:old_thw} have become obsolete
%with the introduction of the partitioned output types \ref{out:first_part}
%through \ref{out:last_part}. However, they are retained in the present model
%release to provide required downward compatibility with model version 1.18 at
%NCEP. These output types are likely to be retired in upcoming versions of \ws.
